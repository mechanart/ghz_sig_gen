\documentclass[letterpaper,12pt]{article}
\usepackage{amsmath}
\title{AD9914 Initialization and Configuration\\ \vspace{2 mm} {\large version 0.1}}
\author{Logan Cummings}
\begin{document}
\maketitle
\renewcommand*\footnoterule{}
\let\thefootnote\relax\footnote{AD9914/GSG \hfill Programming Notes  \hfill {\copyright{2014, Mechanart, LLC}}}
\newpage
\tableofcontents

\section{Introduction}
This is a collection of notes describing various settings and values that
need to be captured in the code that is talking to the DDS. 

\section{Startup}
\subsection{VCO CAL}
If the PLL is used to generate the system clock, the VCO must be calibrated at
start-up. 
\begin{verbatim}
spi_set(0x00[24], 1);
# Check for PLL lock
while (!locked)
{
  spi_get(0x00[24], 0);
  sleep(5);
}
\end{verbatim}


\subsection{DAC CAL}The DAC CAL enable bit must be set and then cleared at start-up and after
the system clock has changed. 
e.g.:
\begin{verbatim}
spi_set(0x03[24], 1);
sleep(t_cal);
spi_set(0x03[24], 0);
\end{verbatim}
Where $t\_cal$ is determined as follows:
\[ 
  t_{cal} = \frac{469,632}{f_S} 
\]

n.b. - the VCO CAL must be performed prior to DAC CAL. 

\end{document}
