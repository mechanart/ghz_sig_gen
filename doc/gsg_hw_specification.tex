\documentclass[letterpaper,12pt]{article}
\usepackage{amsmath}
\usepackage{mathrsfs}
\DeclareMathOperator{\Lph}{\mathscr{L}}
\title{GSG HW Specification \\ \vspace{2 mm} {\large version 0.1}}
\author{Logan Cummings}
\begin{document}
\maketitle
\renewcommand*\footnoterule{}
\let\thefootnote\relax\footnote{GSG \hfill HW Specification  \hfill {\copyright{2014-2016, Mechanart, LLC}}}
\newpage
\tableofcontents

\section{Introduction}

The Gigahertz Signal Generator (GSG) was born out of a need for a very clean 
local oscillator for an high performance HF radio receiver. This project 
has had a long inception, starting with the discovery of the CDG2000 
transceiver project Martein Bakker, PA3AKE's refined H-Mode Front End.
\par 
While Martein had meticulously designed and tested each part of the receiver 
down to the power and audio outputs, he had stopped short of releasing a 
board for the local oscillator. This was due to the fact that the best perfomance 
for full  HF band operation was achieved by a DDS, however the best DDS chips
available at the time (e.g. the AD9910) had issues with amplitude noise.
\par
This project was started in 2013 based on Martein's conclusion that the missing 
DACBP pin was to blame for the poor amplitude noise performance, and began to 
design a local oscillator board around the most recent DDS from Analog Devices, 
the AD9914.
\par
While the AD9914 has only a 12-bit DAC vs. the AD9910's 14-bits, it does bring
out the DACBP pin for decoupling the internal DAC reference. Additionally, it 
can be clocked at nearly 4x the frequency (3.5GHz vs. 1GHz for the AD9910). 
\par
Given those differences, work began to best utilize the AD9914 to produce a 
low phase noise frequency agile frequency synthesizer for general laboratory 
usage, while still meeting the requirements for the high performance LO.
\par
Finally, I found the Easy-Phi project and designed the new board to 
be Easy-Phi compatible, which also means it contains an Arduino (Due) compatible 
microprocessor.

\subsection{Requirements}
The original requirements are presented in the scope of the PA3AKE HF 
front-end.

With a 9MHz IF, Martein's front-end has such good IM3DR that the local oscillator
phase noise must be less than -151dBc/Hz at 1KHz, and less than -156dBc/Hz at 
10KHz so as to not be the limiting factor in the receivers overall dynamic range.

This is the primary requirement for the local oscillator and therefore this signal 
generator shall meet that requirement:
\begin{center}
\begin{tabular}{| l | c | r |}
\hline
\multicolumn{3}{|l|}{Performance Requirement at 16MHz Output} \\ \hline
Phase Noise @ 1kHz Offset & -151 & dBc/Hz \\ \hline
Phase Noise @ 10kHz Offset & -156 & dBc/Hz \\ \hline
Phase Noise @ 100kHz Offset & -156 & dBc/Hz \\ \hline
\end{tabular}
\end{center}

Therefore, if we consider the DDS as a frequency divider of its reference 
clock, then we have the following requirement for the phase noise of the 
reference clock:
\[
  \Lph(f)_{ref} = \Lph(f)_{out} + 20LOG(N)
\]

Where N is the division ratio between the reference clock and output frequency.

Since the higher the reference clock relative to the output frequency the
higher the spur-free dynamic range, we then require a 3.5GHz oscillator 
with 
\begin{align*}  
  \Lph(f)_{ref} &\leq -151 + 20LOG(\frac{3500}{16}) \\
  \Lph(f)_{ref} &\leq -104.2 dBc/Hz
\end{align*}
at 1kHz offset.
\par 
Martein's prototype AD9910 board used a very low noise OCXO at 100MHz 
multiplied by an x5 multiplier followed by a doubler circuit to achieve 
1GHz with good phase noise. After researching x5 and x7 multiplier circuits
to do the same to 3.5GHz, a 3.5GHz CRO was found that nearly met all the 
requirements with much less complexity.
\begin{center}
\begin{tabular}{| l | c | c | c | r |}
\hline
\multicolumn{5}{|c|}{Performance Requirement of 3.5GHz RefClk} \\ \hline 
Clock & Target & ABLNO x35 & CVCO55 3500 & \\ \hline
PN @ 1kHz Offset & -104.2 & -110.1 & -89.1 & dBc/Hz \\ \hline
PN @ 10kHz Offset & -109.2 & -129.1 & -115 & dBc/Hz \\ \hline
PN @ 100kHz Offset & -109.2 & -130.1 & -135 & dBc/Hz \\ \hline
\end{tabular}
\end{center}
However, the above assumes a noiseless multiplier, which is hard to do.
\par
The decision was made to start with the CVCO55 CRO oscillator and prove 
the rest of the DDS signal generator. Once it can be shown the limiting 
factor for close-in phase noise is the reference clock, a multiplied 
crystal oscillator drop-in module for the same 0.55x0.55" footprint can 
be developed.


\section{Connectors and Pinouts}
\subsection{Power Connector}
The component boards designed by PA3AKE use an FCI power part.

\subsection{SMA Output Connector} 
The main RF output from the DAC is routed to an edge-mount SMA connector. 
THis is compatible with the jumper wiring of the PA3AKE modules.

\subsection{Optional External Oscillator Input}
An SMA connector is provided on the component side of the primary board 
for providing an external clock signal. The clock signal must have slew rate 
\[
  SR \geq 2\pi f  Vpk
\]

Zynq Processor FPGA Option
Parallel Interface 
Serial Interface 
Noise Measurements
Anti-aliasing Filter
\end{document}
